\question % Based on Mathematics HL (Core) - Third Edition, RS22C, Q15.
Consider the two functions $f(x) = \cos(x)$ and $g(x) = \sin(x)$. The graphs of $y = f(x)$ and $y = g(x)$ are shown in Figure~\vref{fig:area-between-two-functions}.

\begin{figure}[h]
	\centering
	\begin{tikzpicture}
	\begin{axis}[
		my axis style,
		ticks=none,
		ymax=1.2,
		legend pos=outer north east,
		legend entries={
			$y = f(x)$,
			$y = g(x)$,
		},
	]
	\addplot[
		domain=-.375:2,
		<->,
	] {cos(x)};
	\addplot[
		domain=-.375:2,
		red,
		<->,
	] {sin(x)};
	\addplot+[
		name path=A,
		domain=-0:.7854,
		opacity=0,
	] {cos(x)};
	\addplot+[
		name path=B,
		domain=0:.7854,
		opacity=0,
	] {sin(x)};
	\addplot[
		pattern=north west lines,
		pattern color=blue,
		opacity=.6,
	] fill between[of=A and B];
	\draw (.3972,0) node [] {\AxisRotator[rotate=180, ->, black, densely dotted]};
	\ifprintanswers
	\fill[red] (.7854,.7071) node[right, xshift=2pt] {$P(\frac{\pi}{4}, \frac{1}{\sqrt{2}})$} circle (2pt);
	\else
	\fill[black] (.7854,.7071) node[above] {$P$} circle (2pt);
	\fi
	\end{axis}
	\end{tikzpicture}
	\caption{The shaded region is rotated $\ang{360}$ about the $x$-axis.}
	\label{fig:area-between-two-functions}
\end{figure}

\begin{parts}

\part[2]
Find the coordinates of point $P$.

\begin{EnvFullwidth}
\begin{solutionorgrid}[1.5in]
The graphs meet when
\begin{align*}
	\sin(x) &= \cos(x) \\
	\tan(x) &= 1 && (\cos(x) \neq 0) \\
	x &= \frac{\pi}{4}.
\end{align*}
So $P(\frac{\pi}{4}, \frac{1}{\sqrt{2}})$.
\end{solutionorgrid}
\end{EnvFullwidth}

\part[4]
Find the \emph{exact} volume of the solid of revolution depicted in Figure~\ref{fig:area-between-two-functions}.

\begin{EnvFullwidth}
\begin{solutionorgrid}[3.5in]
The volume is
\begin{align*}
	V &= \pi \int_0^{\sfrac{\pi}{4}} y^2 \dif x \\
	&= \pi \int_0^{\sfrac{\pi}{4}} \cos^2(x) - \sin^2(x) \dif x \\
	&= \pi \int_0^{\sfrac{\pi}{4}} \cos(2x) \dif x \\
	&= \pi \Eval{\frac{1}{2} \sin(2x)}{0}{\sfrac{\pi}{4}} \, \\
	&= \frac{\pi}{2} (\sin(\tfrac{\pi}{2}) - \sin(0)) \\
	&= \frac{\pi}{2} \, \textrm{units}^3.
\end{align*}
\end{solutionorgrid}
\end{EnvFullwidth}

\end{parts}
